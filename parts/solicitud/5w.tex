\section{Las cinco W}

% En esta sección trataremos los siguientes puntos, con la información ofrecida por el Ministerio de Justicia y lo que queramos desarrollar
% - Qué es
% - Quién puede solicitarlo
% - Cómo puede hacerse
% - Cuándo hacerlo
% - Dónde hacerlo
% - Para qué hacerlo
En esta sección queda descrito a continuación información relevante de cara al consumidor por parte de la administración pública, en este caso, el Ministerio de Justicia.

\subsection{¿Qué es?}\label{subsec:what}
Avanzada su definición en la introducción, el \textbf{Certificado de Actos de Última Voluntad} es el documento que acredita si una persona ha otorgado testamento/s y ante qué Notarios/s. De esta forma, las personas a quien pueda corresponder algún derecho de sucesión podrán dirigirse al Notario autorizante del último testamento y obtener una copia oficial del mismo. Es fundamental la posesión de este certificado para trámites posteriores de carácter sucesorio, como la aceptación y partición de herencia, gestión del impuesto de sucesiones, cambios de titularidad, trámites con la Seguridad Social \dots

\subsection{¿Quién puede solicitarlo?}
Cualquier persona interesada en obtener esta certificación y su correspondiente información, con la precondición de que el \textbf{Certificado de defunción} de la persona en cuestión se encuentre formalizado y oficializado en el sistema informático del Ministerio.

\subsection{¿Cómo puede hacerse?}
La solicitud está disponible vía electrónica \cite{solicitud}, por correo postal o vía presencial. 

En el proceso administrativo, el demandante debe abonar la tasa 006 asociadas a la solicitud, proporcionando información de pago a través de pasarelas de pago de la Agencia Tributaria para Tasas Administrativas. 

No todas las entidades bancarias están adheridas a la modalidad, además de tener sus respectivos horarios de tramitación, por lo que es esencial consultar el apartado de más información disponible en la solicitud.

Como justificante de su solicitud se podrá descargar un documento con los comprobantes del pago y del registro de la solicitud, \textbf{firmado electrónicamente por el Ministerio de Justicia}. Es imprescindible el guardado de este volante, pues contiene un número de solicitud que permitirá al consumidor, en un paso posterior, reclamar el certificado.

\subsection{¿Cuándo hacerlo?}
Bajo la información expuesta en el Artículo 30 de la Ley 39/2015, de 1 de octubre, del Procedimiento Administrativo Común de las Administraciones Públicas \cite{LPACAP} (en adelante, ``LPACAP''), la solicitud no podrá presentarse hasta transcurridos 15 días hábiles a contar desde el siguiente a la fecha oficial de fallecimiento, excluyendo sábados, domingos y festivos nacionales y de la Comunidad de Madrid, autonomía donde se ubica el Registro General de Actos de Última Voluntad.

\subsection{¿Dónde hacerlo?}
A través de la Sede Electrónica del Ministerio de Justicia, siempre que el sistema informatizado tenga la información suficiente para proceder a la solicitud. En caso contrario, requiere atención postal o presencial.

\subsection{¿Para qué hacerlo?}
Avanzado en la \subsecref{what}, el certificado es necesario para realizar cualquier trámite sucesorio.
