\section{Reglamento y Normativa}
% En esta sección comentaremos el reglamento y normativa que afecta a la solicitud y debemos tener en cuenta
% - Reglamento Notarial de 2 de junio de 1944, Anexo II.
% - Resolución de 13 de enero de 2011, por la que se determinan los requisitos y condiciones para tramitar por vía telemática los certificados de últimas voluntades y contratos de seguros de cobertura de fallecimiento.

El artículo 9 de la LPACAP \cite{LPACAP} establece que los interesados en realizar un procedimiento administrativo con una entidad pública \textquote{podrán identificarse electrónicamente ante las Administraciones Públicas a través de [\dots] [s]istemas basados en certificados electrónicos cualificados de firma electrónica}.

La Resolución de 13 de enero de 2011, de la Dirección General de los Registros y del Notariado, por la que se determinan los requisitos y condiciones para tramitar por vía telemática las solicitudes de los certificados de últimas voluntades y contratos de seguros de cobertura de fallecimiento y se establecen modificaciones en el Modelo 790 de autoliquidación y de solicitud e instrucciones, para las solicitudes presenciales y por correo de los certificados de actos de última voluntad y contratos de seguros de cobertura de fallecimiento \cite{Resolucion132011} (en adelante, `Resolución 13/2011') especifica los pasos a seguir para la emisión de dichos certificados, y el protocolo que se ha de seguir.

Esta resolución remarca en su título segundo los tipos de certificados admitidos, especificando que \textquote{[l]a identificación y firma del solicitante que inicie los procedimientos requerirá el uso de [s]istemas de firma electrónica incorporados al documento nacional de Identidad, para personas físicas [o s]istemas de firma electrónica avanzada, incluyendo los basados en un certificado electrónico reconocido}.