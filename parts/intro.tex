\part{Introducción}
En España, comienza un tedioso camino jurídico cuando una persona del entorno fallece y entra en duda la herencia y otorgamiento de testamento. El \textbf{Certificado de Actos de Última Voluntad} es un documento fundamental en este ámbito, que permite identificar si el difunto dispone de última disposición, además de conocer el notario donde este se encuentra depositado. Es un requisito indispensable para gestionar la sucesión de bienes y derechos, garantizando el respeto a la resolución del fallecido y facilitando la ejecución de los trámites sucesorios.

Debido a su carácter crítico, la obtención de este certificado representa una etapa clave en el proceso de transmisión patrimonial, en la que deben garantizarse tanto la exactitud como la seguridad jurídica de los datos. Por estos motivos, su análisis resulta esencial para comprender las implicaciones legales y administrativas asociadas, así como para proponer mejoras de tiempo en su gestión.

En las secciones siguientes, queda argumentado en las 5 W (qué, quién, cuándo, dónde y por qué), no solo la naturaleza del certificado, sino también el marco normativo que lo regula, la documentación y requisitos necesarios para su solicitud, y una propuesta concreta de mejora para optimizar su acceso y funcionalidad.
% Aquí coloquemos la introducción, basícamente comentemos que hemos escogido el proceso de certificado de actos de última voluntad, que se ha seleccionado por su completitud y su criticidad