\part{Conclusiones}
El proceso para obtener el \textbf{Certificado de Actos de Última Voluntad} desempeña un papel crucial en los trámites sucesorios en España. Este documento garantiza la seguridad jurídica y facilita la gestión del reparto de bienes y derechos conforme a los deseos del fallecido. Sin embargo, a pesar de su relevancia, el proceso presenta desafíos que afectan la experiencia del usuario y la eficiencia del sistema.

La propuesta de incorporar tecnologías basadas en sistemas de eventos para gestionar la certificación de defunción promete un impacto positivo significativo. Este enfoque permitiría reducir los tiempos de espera de los interesados, con una actualización en tiempo real de los datos en los sistemas informatizados del Ministerio de Justicia. Ofrece una mayor resiliencia y eficiencia y una mejor experiencia para el usuario, eliminando posibles fricciones tecnológicas.

En definitiva, el proceso de solicitud actual cumple con su propósito, con unas indicaciones esenciales de su tramitación, precondiciones y expone claramente la necesidad del mismo. Sin embargo, consideramos que la implementación de estas mejoras no solo beneficiaría a los solicitantes, sino también a la administración pública, haciendo que los trámites sean más efectivos y accesibles para todos.
% Ofrecemos unas conclusiones acerca del proceso del certificado, resumiendo sus procesos y haciendo una referencia a la propuesta de mejora, que genera un impacto positivo en la ciudadanía.